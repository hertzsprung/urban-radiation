\documentclass[a4paper,titlepage]{article}
\usepackage[T1]{fontenc}
\usepackage{charter}
\usepackage[bitstream-charter]{mathdesign}
\usepackage{siunitx}
\usepackage{amsmath}
\usepackage{mathtools}

\begin{document}
\title{Impact of urban atmospheric conditions on radiation receipt in London}
\author{Elizabeth Erhbar, Maurice John Lally, Emily McKie, James Shaw}
\maketitle

The simplest model assumes no reflection or absorption occurs in the atmosphere; that is, transmittance $\mathcal{T} = 1$.  From, TODO: section 9.4 ambaum, the insolation $S$ is related to the total solar irradiance $S_0$ by
\begin{align*}
S &= S_0 \left( \langle r_E \rangle / r_E(t) \right)^2 \cos \theta
\intertext{Neglecting orbital eccentricity this becomes}
S &= S_0 \cos \theta
\intertext{$\theta$ is the solar zenith angle which is approximated by}
\cos \theta &= \sin \varphi \sin \delta + \cos \varphi \cos \delta \cos h
\end{align*}
where $\varphi$ is the latitude, $h$ is the hour angle and $\delta$ is the declination of the Sun.  Given we are concerned with radiation in London, we can assume a longitude of \ang{0}, so the hour angle is approximately
\begin{align*}
h = \left( t(\mathrm{seconds}) \cdot \ang{360} / 86400 \right) - \ang{180}
\end{align*}
The declination of the Sun $\delta$ is approximated by
\begin{align*}
\delta &= \ang{-23.44} \cdot \cos \left[ \frac{\ang{360}}{365} \cdot (N+10) \right]
\end{align*}
where $N$ is the day of the year with $N=0$ being January 1.  This model results in insolation with annual and daily periods as seen in Figure~\ref{fig:toa-model}.

\begin{figure}
\centering
\input{toa-model}
\caption{Insolation model at \SI{51}{\degree N}}
\label{fig:toa-model}
\end{figure}

\end{document}
