\documentclass[a4paper,titlepage, twoside]{report}
\usepackage[T1]{fontenc}
\usepackage{charter}
\usepackage[bitstream-charter]{mathdesign}
\usepackage{siunitx}
\usepackage{amsmath}
\usepackage{mathtools}
\usepackage{subcaption}
\usepackage[backend=biber, style=alphabetic]{biblatex}

\addbibresource{radiation.bib}

\newcommand\eqnumbered{\addtocounter{equation}{1}\tag{\theequation}}
\newcommand\Kdownsfc{{K\!\!\downarrow}_\mathrm{SFC}}
\newcommand\Kdowntoa{{K\!\!\downarrow}_\mathrm{TOA}}
\newcommand\Ldownsfc{{L\!\!\downarrow}_\mathrm{SFC}}

\begin{document}
\title{Impact of urban atmospheric conditions on radiation receipt in London}
\author{Elizabeth Erhbar, Emily McKie, Maurice John Lally, James Shaw}
\maketitle

\begin{abstract}
TODO

A model is developed to predict downwelling longwave and shortwave radiation at the surface using observed cloud cover, air temperature and relative humidity.  Model results have a root-mean-squared error of \SI{112}{\watt\per\meter\squared} for shortwave and \SI{47}{\watt\per\meter\squared} for longwave.
\end{abstract}

\chapter{Model}
The radiation model was constructed around observation data collected from (TODO: MET mast instrument name) from January 2010 to December 2012, and Vaisala CL31 cloud cover data from October to December 2010.  Both instruments are installed at KSS. TODO: footnote explaining where that is

\section{Design}
The model diagnoses shortwave and longwave radiation from three easily observed meteorological variables: air temperature near the surface, cloud cover, and relative humidity.  The shortwave radiation model is driven by the diurnal and seasonal variation of insolation through an atmosphere with a degree of cloud cover.  A simple longwave model is present that is based on air temperature alone, which is then refined based on longwave parameterisation from \cite{loridan}.

\subsection{Geometry}
The model initially estimates the insolation at the top of the atmosphere (TOA).  From \cite{ambaum} section 9.4, the insolation at the top of the atmosphere, $\Kdowntoa$, is related to the total solar irradiance, $S_0$
\begin{align}
\Kdowntoa &= S_0 \left( \langle r_E \rangle / r_E(t) \right)^2 \cos \theta
\intertext{where $\langle r_E \rangle$ is the average distance between the Earth and the Sun and $r_E(t)$ is the distance at a given time.  This orbital eccentricity is neglected since it is small, so the equation simplifies to}
\Kdowntoa &= S_0 \cos \theta
\intertext{where $\theta$ is the solar zenith angle which is approximated by}
\cos \theta &= \sin \varphi \sin \delta + \cos \varphi \cos \delta \cos h % TODO Jacobson p317
\end{align}
where $\varphi$ is the latitude, $h$ is the hour angle and $\delta$ is the declination of the Sun.  Given we are concerned with radiation in London, we can assume a longitude of \ang{0}, so the hour angle is approximately
\begin{align}
h = \left( t(\mathrm{seconds}) \cdot \ang{360} / 86400 \right) - \ang{180}
\end{align}
The declination of the Sun $\delta$ is approximated by
\begin{align}
\delta &= \ang{-23.44} \cdot \cos \left[ \frac{\ang{360}}{365} \cdot (N+10) \right] % TODO: citation
\end{align}
where $N$ is the day of the year with $N=0$ being January 1.  This model results in insolation with annual and diurnal periods as seen in Figure~\ref{fig:toa-model}.

\begin{figure}
\centering
\begin{subfigure}{0.45\textwidth}
\input{toa-model-annual}
\caption{Annual variation of insolation in 2010.  Midday insolation peaks in the summer when London is closest to the Sun.}
\end{subfigure}
\hfill
\begin{subfigure}{0.45\textwidth}
\input{toa-model-daily}
\caption{Insolation on January 1 2010.  Times of day in UTC with sunrise and sunset occuring around 08:00 and 16:00 respectively.}
\end{subfigure}
\caption{Modelled insolation at top of the atmosphere above London (\SI{51}{\degree N}) showing annual and diurnal periodicity.}
\label{fig:toa-model}
\end{figure}

\subsection{Shortwave insolation}
To model the insolation at the surface we estimate the optical depth of the atmosphere at a given time, $\tau(t)$.  This is calculated by comparing insolation at the surface, $\Kdownsfc$, and at the top of the atmosphere, $\Kdowntoa$ using the Beer-Lambert law \cite{stephens}
\begin{align}
&& \Kdownsfc &= \Kdowntoa\: \exp \left( -\frac{\tau}{\mu} \right) \\
\text{or equivalently} && \tau &= \mu \: \ln \left( \frac{\Kdownsfc}{\Kdowntoa} \right)
\end{align}
In Figure~\ref{fig:cloud-tau-fit} we find a weak relationship between observed cloud cover fraction $F_\mathrm{cloud}$ and optical depth $\tau$.  Using a simple linear regression we find
\begin{align}
\tau &= \gamma F_\mathrm{cloud} + \tau_\mathrm{clear}
\end{align}
where $\tau_\mathrm{clear}$ is the optical depth of clear sky ($F_\mathrm{cloud} = 0$) and $\gamma$ is the cloud optical depth coefficient.  At times when cloud coverage observations are unavailable surface insolation is modelled using the mean optical depth $\langle \tau \rangle$.  

\subsection{Longwave radiation}
The downwelling longwave radiation at the surface, $\Ldownsfc$, can be estimated by using a single-slab, isothermal atmosphere.  Using the observed air temperature immediately above the surface and assuming the atmosphere radiates as a blackbody we can use the Stefan--Boltzmann law \cite{ambaum} % TODO p.168
\begin{align}
\Ldownsfc = \varepsilon_\mathrm{ATM} \sigma T_\mathrm{ATM}^4 \label{eq:stefan-boltzmann}
\end{align}
where we assume that the atmospheric emissivity $\varepsilon_\mathrm{ATM} = 1$.  In our analysis in section~\ref{sec:model-analysis} we find that this simple approximation overestimates downwelling longwave radiation.

\cite{loridan} propose a more refined longwave radiation model that is controlled by relative humidity $\mathrm{RH}$, cloud cover fraction $F_\mathrm{cloud}$ as well as air temperature.  Precipitable water content is a function of vapour pressure $e$ and temperature $T_\mathrm{ATM}$
\begin{align}
w &= 46.5 \frac{e(\si{\hecto\pascal})}{T_\mathrm{ATM}(\si{\kelvin})}
\intertext{From \cite{ambaum}, the vapour pressure is a function of relative humidity and saturated vapour pressure $e_s$}
e &= \mathrm{RH} \: e_s(T)
\intertext{where the saturated vapour pressure can be estimated by Tetens' formula}
e_s(\si{\hecto\pascal}) &= 6.112 \exp \left( \frac{17.67 T(\si{\celsius})}{T(\si{\celsius}) + 243.5} \right)
\intertext{The atmospheric emissivity is based on clear-sky emissivity and cloud cover fraction}
\varepsilon_\mathrm{ATM} &= \varepsilon_\mathrm{clear} + \left(1 - \varepsilon_\mathrm{clear} \right) F_\mathrm{cloud} \\
\varepsilon_\mathrm{clear} &= 1 - \left( 1 + w \right) \exp \left( - \sqrt{1.2 + 3w} \right)
\intertext{Substituting into Equation~\ref{eq:stefan-boltzmann}, the parameterisation of $\Ldownsfc$ becomes}
\Ldownsfc &= \left[ \varepsilon_\mathrm{clear} + \left( 1 - \varepsilon_\mathrm{clear} \right) F_\mathrm{cloud} \right] \sigma T_\mathrm{ATM}^4
\end{align}

\section{Parameterization}
TODO: how did we handle SFC SW > TOA SW?  What about other bits of missing data?

TODO: move everything to do with coefficients (anything that depends on observed values) in here
\begin{figure}
\centering
\input{cloud-tau-fit}
\caption{Relation between cloud cover and observed optical depth}
\label{fig:cloud-tau-fit}
\end{figure}

We find that $\langle \tau \rangle = 0.45$, $\tau_\mathrm{clear} = 0.14$,and $\gamma =  0.29$ with a residual sum of squares of 166.
%TODO: gnuplot told me this but my knowledge of stats isn't good enough to properly understand it.

\section{Analysis}
\label{sec:model-analysis}
The model is evaluated by comparing its output with radiation observed at KSS.  Longwave radiation is better approximated but overestimates the observed value.  

As seen in Figure~\ref{fig:shortwave-verification}, the model accurately diagnoses shortwave radiation with cloud cover observations alone with a root-mean-square error (RMSE) of \SI{112}{\watt\per\meter\squared}.  When total cloud cover is observed for an extended duration of about one day, shortwave radiation is overestimated.  An example is shown in Figure~\ref{fig:extended-cloud}.  In section~\ref{sec:further-work} we suggest a way of improving this.

Longwave radiation is systematically overestimated by the simple temperature-only model with a RMSE of \SI{59}{\watt\per\meter\squared}.  The Loridan longwave model is more accurate having a RMSE of \SI{47}{\watt\per\meter\squared}.  The two longwave models are compared with observed radiation in Figure~\ref{fig:longwave-verification} which shows systematic errors in both models.

TODO: This could be due to the model wrongly assuming that air is a blackbody, and ignoring absorption and scattering effects.  Or the coefficients used in Loridan aren't quite right for us?  But I wouldn't know how to recalculate them...
\begin {figure}
\centering
\input{shortwave-verification}
\caption{Comparison of modelled and observed shortwave radiation TODO date range}
\label{fig:shortwave-verification}
\end{figure}

\begin{figure}
\centering
\input{extended-cloud}
\caption{Modelled and observed shortwave radiation with extended cloud cover}
\label{fig:extended-cloud}
\end{figure}

\begin{figure}
\centering
\input{longwave-verification}
\caption{Modelled and observed longwave radiation TODO date range}
\label{fig:longwave-verification}
\end{figure}

\section{Further work}
\label{sec:further-work}
TODO better analysis to distinguish extent of systematic bias/unsystematic error
TODO make shortwave model cope with extended cloud cover -- Loridan says we can guess cloud cover based on RH and temperature
TODO probabalistic model based on uncertainty of SW linear regression
TODO explore aerosols, rainfall


\printbibliography

\end{document}
