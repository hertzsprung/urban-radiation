\documentclass[a4paper,titlepage]{article}
\usepackage[T1]{fontenc}
\usepackage{charter}
\usepackage[bitstream-charter]{mathdesign}
\usepackage{siunitx}
\usepackage{amsmath}
\usepackage{mathtools}

\newcommand\eqnumbered{\addtocounter{equation}{1}\tag{\theequation}}

\begin{document}
\title{Impact of urban atmospheric conditions on radiation receipt in London}
\author{Elizabeth Erhbar, Emily McKie, Maurice John Lally, James Shaw}
\maketitle

Our model starts by estimating the insolation at the top of the atmosphere (TOA).  From, TODO: section 9.4 ambaum, the insolation $S$ is related to the total solar irradiance $S_0$ by
\begin{align*}
S &= S_0 \left( \langle r_E \rangle / r_E(t) \right)^2 \cos \theta
\intertext{Neglecting orbital eccentricity this becomes}
S &= S_0 \cos \theta
\intertext{$\theta$ is the solar zenith angle which is approximated by}
\cos \theta &= \sin \varphi \sin \delta + \cos \varphi \cos \delta \cos h
\end{align*}
where $\varphi$ is the latitude, $h$ is the hour angle and $\delta$ is the declination of the Sun.  Given we are concerned with radiation in London, we can assume a longitude of \ang{0}, so the hour angle is approximately
\begin{align*}
h = \left( t(\mathrm{seconds}) \cdot \ang{360} / 86400 \right) - \ang{180}
\end{align*}
The declination of the Sun $\delta$ is approximated by
\begin{align*}
\delta &= \ang{-23.44} \cdot \cos \left[ \frac{\ang{360}}{365} \cdot (N+10) \right]
\end{align*}
where $N$ is the day of the year with $N=0$ being January 1.  This model results in insolation with annual and daily periods as seen in Figure~\ref{fig:toa-model}.

To approximate the insolation at the surface we compare insolation at the surface with insolation at the top of the atmosphere to calculate an average optical depth using the Beer-Lambert law
\begin{align*}
&& S_\mathrm{SFC} &= S_\mathrm{TOA}\: \exp \left( -\frac{\tau}{\mu} \right) \\
\text{or equivalently} && \tau &= \mu \: \ln \left( \frac{S_\mathrm{SFC}}{S_\mathrm{TOA}} \right)
\end{align*}
We find the average optical depth $\langle \tau \rangle = 0.45$.

The downwelling longwave radiation at the surface can be estimated by assuming that the air emits black body radiation, hence
\begin{align*}
I_{LW} &= \sigma T^4
\end{align*}

We can verify the model by comparing its output with radiation observed at KSS (TODO footnote for what KSS means).  In Figure~\ref{fig:toa-model-verification} we see that the model gives a crude estimate of surface insolation because the optical depth is not constant ($\sigma=0.39$).  Longwave radiation is better approximated but overestimates the observed value.  This could be due to the model wrongly assuming that air is a blackbody, and ignoring absorption and scattering effects.

\begin{figure}
\centering
\input{toa-model}
\caption{Insolation model at \SI{51}{\degree N}}
\label{fig:toa-model}
\end{figure}

\begin{figure}
\centering
\input{toa-model-verification}
\caption{Comparison of modelled and observed radiation}
\label{fig:toa-model-verification}
\end{figure}
\end{document}
