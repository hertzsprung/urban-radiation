\documentclass[a4paper,titlepage, twoside]{report}
\usepackage[T1]{fontenc}
\usepackage{charter}
\usepackage[bitstream-charter]{mathdesign}
\usepackage{siunitx}
\usepackage{amsmath}
\usepackage{mathtools}
\usepackage{subcaption}
\usepackage[backend=biber, style=authoryear, firstinits=true]{biblatex}
\usepackage{minted}
\usepackage{booktabs}

\renewbibmacro{in:}{}

\addbibresource{radiation.bib}

\newcommand\eqnumbered{\addtocounter{equation}{1}\tag{\theequation}}
\newcommand\Kdownsfc{{K\!\!\downarrow}_\mathrm{SFC}}
\newcommand\Kdowntoa{{K\!\!\downarrow}_\mathrm{TOA}}
\newcommand\Ldownsfc{{L\!\!\downarrow}_\mathrm{SFC}}

\begin{document}
\expandafter\def\csname PY@tok@err\endcsname{}

\title{Impact of urban atmospheric conditions on radiation receipt in London}
\author{Elizabeth Erhbar, Emily McKie, Maurice John Lally, James Shaw}
\maketitle

\begin{abstract}
TODO

A model is developed to predict downwelling longwave and shortwave radiation at the surface using observed cloud cover, air temperature and relative humidity.  Model results have a root-mean-squared error of \SI{112}{\watt\per\meter\squared} for shortwave and \SI{47}{\watt\per\meter\squared} for longwave.
\end{abstract}

\tableofcontents

\chapter{Model}
The radiation model was constructed around observation data collected from a Met mast from January 2010 to December 2012, and Vaisala CL31 cloud cover data from October to December 2010.  Both instruments were installed at KSS. TODO: footnote explaining where that is

\section{Design}
The model diagnoses shortwave and longwave radiation from three easily observed meteorological variables: air temperature near the surface, cloud cover, and relative humidity.  The shortwave radiation model is driven by the diurnal and seasonal variation of insolation through an atmosphere with an amount of cloud cover.  In section~\ref{sec:longwave-model} a simple longwave model is presented that is based on air temperature alone, which is then refined based on longwave parameterisation from \cite{loridan}.

\subsection{Geometry}
The model initially estimates the insolation at the top of the atmosphere (TOA).  The insolation at the top of the atmosphere, $\Kdowntoa$, is related to the total solar irradiance, $S_0$ \parencite[p. 175]{ambaum}
\begin{align}
\Kdowntoa &= S_0 \left( \langle r_E \rangle / r_E(t) \right)^2 \cos \theta
\intertext{where $\langle r_E \rangle$ is the average distance between the Earth and the Sun and $r_E(t)$ is the distance at a given time.  This orbital eccentricity is neglected since it is small, so the equation simplifies to}
\Kdowntoa &= S_0 \cos \theta
\intertext{where $\theta$ is the solar zenith angle which is given by \cite[p. 317]{jacobson}}
\cos \theta &= \sin \varphi \sin \delta + \cos \varphi \cos \delta \cos h
\end{align}
where $\varphi$ is the latitude, $h$ is the hour angle and $\delta$ is the declination of the Sun.  Given we are concerned with radiation in London, we can assume a longitude of \ang{0}, so the hour angle is approximately \parencite[p. 319]{jacobson}
\begin{align}
h = \left( t(\mathrm{seconds}) \cdot \ang{360} / 86400 \right) - \ang{180}
\end{align}
The declination of the Sun $\delta$ is approximated by
\begin{align}
\delta &= \ang{-23.44} \cdot \cos \left[ \frac{\ang{360}}{365} \cdot (N+10) \right] % TODO: citation
\end{align}
where $N$ is the day of the year with $N=0$ being January 1.  This model results in insolation with annual and diurnal periods as seen in Figure~\ref{fig:toa-model}.

\begin{figure}
\centering
\begin{subfigure}{0.45\textwidth}
\input{toa-model-annual}
\caption{Annual variation of insolation in 2010.  Midday insolation peaks in the summer when London is closest to the Sun.}
\end{subfigure}
\hfill
\begin{subfigure}{0.45\textwidth}
\input{toa-model-daily}
\caption{Insolation on January 1 2010.  Times of day in UTC with sunrise and sunset occuring around 08:00 and 16:00 respectively.}
\end{subfigure}
\caption{Modelled insolation at top of the atmosphere above London (\SI{51}{\degree N}) showing annual and diurnal periodicity.}
\label{fig:toa-model}
\end{figure}

\subsection{Shortwave insolation}
To model the insolation at the surface we estimate the optical depth of the atmosphere at a given time, $\tau(t)$.  This is calculated by comparing insolation at the surface, $\Kdownsfc$, and at the top of the atmosphere, $\Kdowntoa$ using the Beer-Lambert law \parencite{stephens}
\begin{align}
&& \Kdownsfc &= \Kdowntoa\: \exp \left( -\frac{\tau}{\mu} \right) \\
\text{or equivalently} && \tau &= \mu \: \ln \left( \frac{\Kdownsfc}{\Kdowntoa} \right)
\end{align}
The model treats optical depth as a linear function of cloud cover fraction $F_\mathrm{cloud}$ such that 
\begin{align}
\tau &= \gamma F_\mathrm{cloud} + \tau_\mathrm{clear} \label{eq:linear-sw-cloud}
\end{align}
where $\tau_\mathrm{clear}$ is the optical depth of clear sky ($F_\mathrm{cloud} = 0$) and $\gamma$ is the cloud optical depth coefficient.   Values for $\gamma$ and $\tau_\mathrm{clear}$ are found empirically in section~\ref{sec:parameterisation}.

\subsection{Longwave radiation}
\label{sec:longwave-model}
The downwelling longwave radiation at the surface, $\Ldownsfc$, can be estimated by using a single-slab, isothermal atmosphere.  Using the observed air temperature immediately above the surface and assuming the atmosphere radiates as a blackbody we can use the Stefan--Boltzmann law \parencite[p. 168]{ambaum}
\begin{align}
\Ldownsfc = \varepsilon_\mathrm{ATM} \sigma T_\mathrm{ATM}^4 \label{eq:stefan-boltzmann}
\end{align}
if we assume that the atmospheric emissivity $\varepsilon_\mathrm{ATM} = 1$.  In our analysis in section~\ref{sec:model-analysis} we find that this simple approximation overestimates downwelling longwave radiation.

\cite{loridan} propose a more refined longwave radiation model that is controlled by relative humidity $\mathrm{RH}$, cloud cover fraction $F_\mathrm{cloud}$ as well as air temperature.  Precipitable water content is a function of vapour pressure $e$ and temperature $T_\mathrm{ATM}$ \parencite{loridan}
\begin{align}
w &= 46.5 \frac{e(\si{\hecto\pascal})}{T_\mathrm{ATM}(\si{\kelvin})} \label{eq:precip-water-vapour}
\intertext{The vapour pressure is a function of relative humidity and saturated vapour pressure $e_s$ \parencite{ambaum}}
e &= \mathrm{RH} \: e_s(T)
\intertext{where the saturated vapour pressure can be estimated by Tetens' formula \parencite{ambaum}}
e_s(\si{\hecto\pascal}) &= 6.112 \exp \left( \frac{17.67 T(\si{\celsius})}{T(\si{\celsius}) + 243.5} \right)
\intertext{The atmospheric emissivity is based on clear-sky emissivity and cloud cover fraction \parencite{loridan}}
\varepsilon_\mathrm{ATM} &= \varepsilon_\mathrm{clear} + \left(1 - \varepsilon_\mathrm{clear} \right) F_\mathrm{cloud} \label{eq:emissive-atm} \\
\varepsilon_\mathrm{clear} &= 1 - \left( 1 + w \right) \exp \left( - \sqrt{1.2 + 3w} \right) \label{eq:clear-sky-emissivity}
\intertext{Substituting into Equation~\ref{eq:stefan-boltzmann}, the parameterisation of $\Ldownsfc$ becomes \parencite{loridan}}
\Ldownsfc &= \left[ \varepsilon_\mathrm{clear} + \left( 1 - \varepsilon_\mathrm{clear} \right) F_\mathrm{cloud} \right] \sigma T_\mathrm{ATM}^4
\end{align}
To gain physical intuition, the absorptivity of the atmosphere increases with cloud cover and, by Kirchhoff's law \parencite[p. 166]{ambaum}, so does the emissivity $\varepsilon_\mathrm{ATM}$ (equation~\ref{eq:emissive-atm}).  In clear skies, the absorptivity increases with water vapour content since water vapour is a moderate absorber of longwave radiation.  Figure~\ref{fig:emissivity-response} shows the relation between emissivity, relative humidity and cloud cover fraction.

\begin{figure}
\centering
\begin{subfigure}{0.48\textwidth}
\input{clear-emissivity}
\caption{Emissivity response to relative humidity at $T_\mathrm{ATM}$ \SIrange{0}{20}{\celsius} in clear sky conditions from equation~\ref{eq:clear-sky-emissivity}.  Warmer temperatures cause increased absorptivity and emissivity.}
\end{subfigure}
\hfill
\begin{subfigure}{0.48\textwidth}
\input{cloud-emissivity}
\caption{Emissivity response to cloud cover at $T_\mathrm{ATM} = \SI{10}{\celsius}$ and RH from 0.5 to 1.0 given by equation~\ref{eq:emissive-atm}. Cloud cover fraction has a greater effect at lower relative humidities.}
\end{subfigure}
\caption{Emissivity responses to cloud cover and relative humidity}
\label{fig:emissivity-response}
\end{figure}

\subsection{Data cleansing}
The model must handle records with missing or nonphysical values that are present in the KSS dataset.  

Optical depth is not calculated in two cases.  First, for some records, the observed $\Kdownsfc$ is greater than the calculated $\Kdowntoa$.  This typically occurs at night when $\Kdowntoa$ is always zero but, due to instrument error, $\Kdownsfc$ is a small non-zero value.  Second, optical depth is effectively infinite when $\Kdownsfc = 0$ so it is not calculated.

At times when cloud coverage observations are unavailable this affects modelled shortwave and longwave radiation.  Surface insolation is modelled using the mean optical depth $\langle \tau \rangle$.  Cloud cover is assumed to be zero in the Loridan longwave model.

\section{Parameterisation}
\label{sec:parameterisation}

\begin{figure}
\centering
\input{cloud-tau-fit}
\caption{Relation between cloud cover and observed optical depth.  For the KSS dataset shown, 3085 records exist for which both cloud cover and optical depth are present.}
\label{fig:cloud-tau-fit}
\end{figure}

In Figure~\ref{fig:cloud-tau-fit} we find a weak relationship between observed cloud cover fraction $F_\mathrm{cloud}$ and optical depth $\tau$.  Using a simple linear regression we find coefficients for equation~\ref{eq:linear-sw-cloud} where $\tau_\mathrm{clear} = 0.14$ and $\gamma =  0.29$.

After data cleansing, the average optical depth was found to be $\langle \tau \rangle = 0.45$.  The value is only used twice since, in the KSS dataset, there are only two records without cloud coverage data.

Coefficients for precipitable water content $w$ and clear sky emissivity $\varepsilon_\mathrm{clear}$ were taken directly from \cite{loridan} (see equations~\ref{eq:precip-water-vapour} and \ref{eq:clear-sky-emissivity} respectively).  In \cite{loridan}, cloud cover is approximated from relative humidity and air temperature values.  However, since the CL31 dataset has precomputed cloud coverage values, we use these in model calculations.

\section{Analysis}
\label{sec:model-analysis}
The model was evaluated by comparing its output with radiation observed at KSS.  Longwave radiation is better approximated but overestimates the observed value.  

As seen in Figure~\ref{fig:shortwave-verification}, the model diagnoses shortwave radiation with cloud cover observations alone with a root-mean-square error (RMSE) of \SI{112}{\watt\per\meter\squared}.  Since $R^2 = 0.2$ for the linear regression between cloud cover and optical depth, this error should be expected.

When total cloud cover is observed for an extended duration of about one day, shortwave radiation is overestimated.  An example is shown in Figure~\ref{fig:extended-cloud}.  In section~\ref{sec:further-work} we suggest a way of improving this.

\begin{table}[!b]
\centering
\begin{tabular}{ r @{\hspace{2em}} c c c c }
\toprule
Longwave model &	RMSE (\si{\watt\per\square\meter}) &	$\langle \Ldownsfc^\mathrm{model} \rangle - \langle \Ldownsfc^\mathrm{obs} \rangle$ (\si{\watt\per\square\meter}) \\ \midrule
Temperature-only &	59 &	 				\num[retain-explicit-plus]{+47} \\
Loridan &		47 &					\num{-32} \\ \bottomrule
\end{tabular}
\caption{Summary of longwave model errors.  $\langle \Ldownsfc^\mathrm{model} \rangle$ is the mean model value, $\langle \Ldownsfc^\mathrm{obs} \rangle$ the mean observed value.  The Loridan model has lower RMSE and has a mean is closer to the observed mean.}
\label{tab:longwave-error}
\end{table}

Longwave radiation is systematically overestimated by the simple temperature-only model.  This is clearly seen in Figure~\ref{fig:longwave-verification}.  The simple longwave model has a RMSE of \SI{59}{\watt\per\meter\squared}.  The Loridan longwave model is more accurate having a RMSE of \SI{47}{\watt\per\meter\squared}.  The two longwave model errors are summarized in Table~\ref{tab:longwave-error}.

There are several possibilities for model error in longwave radiation.  First, both models assume an isothermal atmosphere and use air temperature at the surface to estimate $\Ldownsfc$.  They both assume that air radiates as a black body.  Second, aerosol concentration is not included in either model; possible aerosol effects are discussed in section~\ref{sec:further-work}.  Third, recalculating the coefficients for the Loridan model using on the KSS dataset may reduce model error (see equations~\ref{eq:precip-water-vapour} and \ref{eq:clear-sky-emissivity}).
\begin {figure}
\centering
\input{shortwave-verification}
\caption{Comparison of modelled and observed shortwave radiation from October 23 -- October 26, 2010}
\label{fig:shortwave-verification}
\end{figure}

\begin{figure}
\centering
\input{extended-cloud}
\caption{Modelled and observed shortwave radiation with extended cloud cover}
\label{fig:extended-cloud}
\end{figure}

\begin{figure}
\centering
\input{longwave-verification}
\caption{Modelled and observed longwave radiation from October 27 -- October 30, 2010.  It is seen that the Loridan model better captures variation in longwave radiation.}
\label{fig:longwave-verification}
\end{figure}

\section{Further work}
\label{sec:further-work}
Our analysis has neglected observation error and uncertainty in cloud optical depth coefficient $\gamma$.  Ideally, our model would incorporate these sources of error and model radiation as a probability distribution rather than a single value.

The shortwave model presented is a crude one.  A selection of superior cloud cover shortwave models are described in \cite{ingram}.  The shortwave model could also be improved by including rayleigh scattering, water vapour and aerosol extinction effects.

Additionally, it may be worthwhile exploring the rainfall measurements present in the KSS dataset since precipitation is known to reduce aerosol concentrations due to washout. TODO citation. 

\printbibliography

\appendix
\chapter{Model source code}
\vspace{-2em}
Source code is available online at \url{https://github.com/hertzsprung/urban-radiation}.

\section{\texttt{radiation.py}}
\inputminted[fontsize=\footnotesize, tabsize=4]{python}{radiation.py}

\section{\texttt{one\_year\_toa.py}}
\inputminted[fontsize=\footnotesize, tabsize=4]{python}{one_year_toa.py}

\section{\texttt{model\_obs\_merge.py}}
\inputminted[fontsize=\footnotesize, tabsize=4]{python}{model_obs_merge.py}

\end{document}
